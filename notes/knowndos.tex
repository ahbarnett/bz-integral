\documentclass[11pt]{article}

\textwidth 6.5in
\oddsidemargin=0in
\evensidemargin=0in
\textheight 9in
\topmargin -0.5in

\usepackage{graphicx,bm,amssymb,amsmath,amsthm}
\usepackage{hyperref}

%\usepackage{showlabels}

\newcommand{\bi}{\begin{itemize}}
\newcommand{\ei}{\end{itemize}}
\newcommand{\ben}{\begin{enumerate}}
\newcommand{\een}{\end{enumerate}}
\newcommand{\be}{\begin{equation}}
\newcommand{\ee}{\end{equation}}
\newcommand{\bea}{\begin{eqnarray}} 
\newcommand{\eea}{\end{eqnarray}}
\newcommand{\ba}{\begin{align}} 
\newcommand{\ea}{\end{align}}
\newcommand{\bse}{\begin{subequations}} 
\newcommand{\ese}{\end{subequations}}
\newcommand{\bc}{\begin{center}}
\newcommand{\ec}{\end{center}}
\newcommand{\bfi}{\begin{figure}}
\newcommand{\efi}{\end{figure}}
\newcommand{\ca}[2]{\caption{#1 \label{#2}}}
\newcommand{\ig}[2]{\includegraphics[#1]{#2}}
\newcommand{\tbox}[1]{{\mbox{\tiny #1}}}
\newcommand{\mbf}[1]{{\mathbf #1}}
\newcommand{\half}{\mbox{\small $\frac{1}{2}$}}
\newcommand{\vt}[2]{\left[\begin{array}{r}#1\\#2\end{array}\right]} % 2-col-vec
\newcommand{\mt}[4]{\left[\begin{array}{rr}#1&#2\\#3&#4\end{array}\right]} % 2x2
\newcommand{\eps}{\varepsilon}
\newcommand{\bigO}{{\mathcal O}}
\newcommand{\sfrac}[2]{\mbox{\small $\frac{#1}{#2}$}}
\newcommand{\R}{\mathbb{R}}
\newcommand{\C}{\mathbb{C}}
\newcommand{\N}{\mathbb{N}}
\newcommand{\Z}{\mathbb{Z}}
\DeclareMathOperator{\re}{Re}
\DeclareMathOperator{\im}{Im}
\DeclareMathOperator{\tr}{Tr}
\DeclareMathOperator{\res}{res}
\newtheorem{thm}{Theorem}
\newtheorem{lem}[thm]{Lemma}
\newtheorem{alg}[thm]{Algorithm}
\newtheorem{pro}[thm]{Proposition}
\newtheorem{cor}[thm]{Corollary}
\newtheorem{rmk}[thm]{Remark}
\newtheorem{conj}[thm]{Conjecture}
% this work...
\newcommand{\om}{\omega}
\newcommand{\tH}{\tilde H}
\newcommand{\BZ}{\Omega}
\newcommand{\kk}{\mbf{k}}
\newcommand{\mm}{\mbf{m}}
\newcommand{\al}{\alpha}
\newcommand{\bt}{\beta}


\begin{document}
\title{Some tight-binding Greens functions with known formulae}
\author{Alex Barnett}
\date{\today}
\maketitle
\begin{abstract}
  We collect known Greens functions (hence density of states)
  formulae for use as validation
  of numerical Brillouin zone integration methods, and in understanding
  the types of complex-plane singularities that occur.
\end{abstract}

Consider a crystalline quantum system with band Hamiltonian $H(\kk)$,
with wavevector $\kk\in \BZ := [0,2\pi)^d$, where the dimension $d=1$, $2$, or $3$.
We are handed a band Hamiltonian function in the form of a finite Fourier
series labelled by an integer frequency $\mm := (m_1,\dots,m_d)$,
\be
H(k) = \sum_{\mm\in{\cal M}} H_m e^{i\mm\cdot \kk}
\label{Hk}
\ee
where ${\cal M} := \{(m_1,\dots,m_d): |m_i|\le M\}$ is the array of 
$(2M+1)^d$ frequency indices centered about the origin.
$H$, and each $H_\mm$, is an $n\times n$ matrix.
$H(\kk)$ is self-adjoint for all $\kk\in\R^d$,
equivalent to the symmetry $H_\mm^* = H_{-\mm}$ for all $\mm\in\Z^d$, where
$^*$ indicates the complex transpose (Hermitian conjugate).

We stick to the paradigm self-energy term $\Sigma = iI\eta$
with broadening parameter $\eta\in\R$.
In this case, $\eta$ can be viewed as the imaginary part of
a complex energy $\om = \om_0 + i\eta$ in the closed upper half plane.
Given the matrix-valued function $H(\kk)$
described by Fourier coefficients $\{H_\mm\}$ up to frequency magnitude $M$,
energy $\om_0$, and $\eta\ge0$,
the quantity (``Green's function'') to compute is
\be
G(\om) \;:=\;
\int_\BZ \tr\, [\om I - H(\kk)]^{-1} d\kk.
\label{G}
\ee
This is a scalar function, analytic in the open upper half plane,
which follows since eigenvalues of $H(\kk)$ can only be real, and by
using that the integral of analytic functions is analytic (via Morera's theorem).
% *** lemma proving this would be good
For $\eta=0$ the limit $\eta=0^+$ is assumed.

We are interested in closed forms for $G(\om)$, to
test numerical methods and to understand
its singularities for $\om$ in the closed upper half plane.
The density of states (DOS) is then
\be
A(\om_0,\eta) := -\frac{1}{\pi}\im G(\om_0+i\eta),
\label{DOS}
\ee
which is thus harmonic, but no longer analytic, with respect to $\om=\om_0+i\eta$ in the upper half plane.
$A$ is usually sought as a function of energy $\om_0$,
fixing some broadening $\eta\ge0$.


\section{Scalar linear chain, square, and simple cubic lattices}

Here $n=1$ (a scalar case). We take fixed
nearest-neighbor tight binding interactions (of strength $1/2$), which become
the coefficients at frequencies $\pm 1$ in each dimension.
Thus $M=1$ is sufficient.
Some literature has
a factor of 2 or $2t$ in the definition (for interaction strength $1$ or $t$) relative to the below.
These results are adapted from \cite[Ch.~5]{economou}
and various Japanese papers of the 1970s, which consider $\om$ real.
In contrast, we treat $\om$ as a complex number in the closed
upper half plane.
We use $G_d(\om)$ for \eqref{G} in dimension $d$.


\bfi % ffffffffffffffffffff
(a)\ig{width=2.8in}{zpaths}    % or use zpaths_jl.png
%\hfill
(b)\ig{width=3.2in}{1dband_cplane}
\ca{Left: Paths of the two roots $z_\pm$ in Section~\ref{s:1d} as
  $\om$ passes along a constant-imaginary-part
  path, passing from very roughly $-2+0.1i$ to $2+0.1i$.
  Right: color plot of complex-valued $G_1(\om)$ in the upper half plane
  (pink has phase zero and yellow has phase $-\pi/2$).
}{f:1d}
\efi

\subsection{$d=1$ linear chain}
\label{s:1d}

We use $x$ for $\kk$ since it is scalar.
The nonzero coefficients are $H_{\pm 1} = 1/2$,
thus $H(x) = \cos x$,
so \eqref{G} becomes
\bea
G_1(\om) &=& \int_0^{2\pi} \frac{dx}{\om - \cos x}
= 2i \int_{|z|=1} \frac{dz}{z^2-2\om z + 1}
= 2i \int_{|z|=1} \frac{dz}{(z-z_+)(z-z_-)},
\label{G1def}
\eea
where we changed variable $z = e^{ix}$ so $dz/z = i\,dx$.
The denominator roots are $z_\pm = \om\pm i\sqrt{1-\om^2}$, where
the sign of the discriminant has been flipped in order to allow
the usual branch cut and square-root definition to apply.
Thus for $\im \om > 0$ the sign of $\im z_\pm$
matches the subscript of $z_\pm$.
Figure~\ref{f:1d}(a) shows the resulting paths of $z_\pm$ as $\re \om$
varies at constant $\im \om$.
For $\im \om \ge 0$ then only $z_-$ is inside the unit circle, so
the residue is $2i/(z_--z_+) = -1/\sqrt{1-\om^2}$, so, by
the residue theorem,
\be
G_1(\om) \;=\; \frac{2\pi}{i \sqrt{1-\om^2}}.
\label{G1}
\ee
Considering $\om$ real ($\eta=0^+$),
$G_1$ is pure negative real for $\om<-1$ (below the band),
pure negative imaginary for $\om \in (-1,1)$ (in the band),
then pure positive real for $\om>1$ (above the band).
See \cite[Fig.~5.6]{economou}.
From \eqref{DOS} the DOS is $A(\om) = 2/\sqrt{1-\om^2}$ in the band and zero
outside the band.
For $\im \om$ small and positive (broadening),
the phase of $G_1(\om)$ rotates rapidly from negative real to negative imaginary as
$\re\om$ increases through $-1$, then continues to rotate to positive real
around $1$.
Here the peak magnitudes are roughly $\sqrt{2/\eta}$.
For $|\om|\gg 1$, $G_1 \sim 2\pi/\om$ decays as if from
a single pole (which the two inverse-sqrt singularities combine to
behave like at large distances). See Fig.~\ref{f:1d}(b).


\subsection{$d=2$ square lattice}
\label{s:2d}

We write $\kk=(x,y)$, and have $H(x,y) = \cos x + \cos y$. The band edges
are thus $\om = \pm 2$.
Economou's approach (clearer than that of Morita 1971) uses the trick that
$H$ is separable when rotated by $\pi/4$; this allows reuse of the $d=1$
case except with a coupling depending on the other variable.
We apply the cosine sum formula with $\al = (x+y)/2$,
$\bt = (x-y)/2$, and use $dx\,dy = 2 d\al\,d\bt$, with domain
$(\al,\bt) \in [0,2\pi)\times[0,\pi)$ which covers $\Omega$ exactly once,
to get
\be
G_2(\om) = \int_0^{2\pi} \int_0^{2\pi} \frac{dx\,dy}{\om - (\cos x + \cos y)}
=
2 \int_0^{\pi} \left[ \int_0^{2\pi} \frac{d\al}{\om - 2\cos\bt\cos\al} \right]
d\bt.
\ee
Writing $\om=(2\cos\bt)\om'$, the inner integral in square brackets
is, reusing \eqref{G1def}--\eqref{G1},
\[
\frac{1}{2\cos\bt}G_1(\om') = 
\frac{1}{2\cos\bt}G_1(\om/2\cos\bt) =
\frac{2\pi}{i}\frac{1}{\sqrt{4\cos^2\bt - \om^2}},
\]
which is analytic in the upper half plane for $\om$.

Consider real $\om>2$ (taking the limit $\im \om = 0^+$).
This limit approaches the sqrt cut from below, so that
$\sqrt{4\cos^2\bt - \om^2} = -i\sqrt{\om^2 - 4\cos^2\bt}$, and
setting $k=2/\om$, we get
\be
G_2(\om) = 2 \frac{2\pi}{i} \int_0^{\pi} \frac{d\bt}{-i \om \sqrt{1-k^2\cos^2\bt}}
= \frac{8\pi}{\om}K(2/\om),
\label{G2re}
\ee
which is real for $\om>2$.
Here the complete elliptic integral is defined for ``modulus'' $k$ by
\be
K(k) := \int_0^{\pi/2} \frac{d\theta}{\sqrt{1-k^2\sin^2 \theta}},
\qquad k\in\C, \; k\neq \pm 1,
\label{ellipk}
\ee
although some definitions and software use $m=k^2$ as the argument.
$K$ has log singularities at $k=\pm 1$, while $K(0) = \pi/2$
means that $G_2 \sim 4\pi^2/\om$ for $\om\gg 1$.
Now, for $\om<-2$, the sqrt cut is approached from above,
but the sign of $\om$ accounts for this, so \eqref{G2re} also holds.
Since $K(-k) = K(k)$, we see $G$ is antisymmetric for real $\om\notin[-2,2]$.
The DOS $A$
vanishes for real $\om$ outside the band $[-2,2]$, by the reality of $G_2$.

The standard branch cut of $K$ is $k\in(-\infty,-1)\cup (1,\infty)$,
following from the standard
branch cut for sqrt in \eqref{ellipk}.
%Along this branch only $\im K$ jumps.
Thus $K$ is analytic for $k$ in the open lower half plane,
equivalent to $\om = 2/k$ in the open upper half plane.
By unique continuation from the real axis outside $[-2,2]$,
\eqref{G2re} holds for all $\im \om >0$.

Finally we tackle the limit of $\om$ approaching $(-2,2)$ from above.
We use the Legendre connection formula (DLMF, 19.7.3)
\be
K(1/k) = k\big( K(k) - iK(k') \big)
\label{Kconn}
\ee
where $k'=\sqrt{1-k^2}$ is the complementary modulus, and now set $k=\om/2$
(the reciprocal of before),
with the imaginary sign taken for $\im k^2 > 0$, appropriate
for $\om$ in the first quadrant.
Thus
\[
G_2(\om) = 4\pi \bigl[ K(\om/2) -i K(\sqrt{1-(\om/2)^2}) \bigr],
\qquad \om \in (0,2),
\]
showing the split into real and imaginary parts.
Here $A_2(\om_0,0) = 4K(\sqrt{1-(\om_0/2)^2})$,
and the DOS drops from a left-sided limit of $2\pi$ to $0$ at $\om_0=2$.
At $\om=0$ there is a log singularity, causing a log blow-up
in $A_2$ at the origin (associated with the flat band structure
around the saddle point $x=y=0$ with $H=0$).
The limit for $\om \in (-2,0)$ flips the sign for the imaginary part in 
\eqref{Kconn}.

*** however DOS has to be positive, check sign in this last case.


\bibliographystyle{abbrv}
\bibliography{refs}

\end{document}
