\documentclass[11pt]{article}

\textwidth 6.5in
\oddsidemargin=0in
\evensidemargin=0in
\textheight 9in
\topmargin -0.5in

\usepackage{graphicx,bm,amssymb,amsmath,amsthm}
\usepackage{hyperref}

%\usepackage{showlabels}

\newcommand{\bi}{\begin{itemize}}
\newcommand{\ei}{\end{itemize}}
\newcommand{\ben}{\begin{enumerate}}
\newcommand{\een}{\end{enumerate}}
\newcommand{\be}{\begin{equation}}
\newcommand{\ee}{\end{equation}}
\newcommand{\bea}{\begin{eqnarray}} 
\newcommand{\eea}{\end{eqnarray}}
\newcommand{\ba}{\begin{align}} 
\newcommand{\ea}{\end{align}}
\newcommand{\bse}{\begin{subequations}} 
\newcommand{\ese}{\end{subequations}}
\newcommand{\bc}{\begin{center}}
\newcommand{\ec}{\end{center}}
\newcommand{\bfi}{\begin{figure}}
\newcommand{\efi}{\end{figure}}
\newcommand{\ca}[2]{\caption{#1 \label{#2}}}
\newcommand{\ig}[2]{\includegraphics[#1]{#2}}
\newcommand{\tbox}[1]{{\mbox{\tiny #1}}}
\newcommand{\mbf}[1]{{\mathbf #1}}
\newcommand{\half}{\mbox{\small $\frac{1}{2}$}}
\newcommand{\vt}[2]{\left[\begin{array}{r}#1\\#2\end{array}\right]} % 2-col-vec
\newcommand{\mt}[4]{\left[\begin{array}{rr}#1&#2\\#3&#4\end{array}\right]} % 2x2
\newcommand{\eps}{\varepsilon}
\newcommand{\bigO}{{\mathcal O}}
\newcommand{\sfrac}[2]{\mbox{\small $\frac{#1}{#2}$}}
\newcommand{\R}{\mathbb{R}}
\newcommand{\C}{\mathbb{C}}
\newcommand{\N}{\mathbb{N}}
\newcommand{\Z}{\mathbb{Z}}
\DeclareMathOperator{\re}{Re}
\DeclareMathOperator{\im}{Im}
\DeclareMathOperator{\tr}{Tr}
\DeclareMathOperator{\res}{res}
\newtheorem{thm}{Theorem}
\newtheorem{lem}[thm]{Lemma}
\newtheorem{alg}[thm]{Algorithm}
\newtheorem{pro}[thm]{Proposition}
\newtheorem{cor}[thm]{Corollary}
\newtheorem{rmk}[thm]{Remark}
\newtheorem{conj}[thm]{Conjecture}
% this work...
\newcommand{\om}{\omega}
\newcommand{\tH}{\tilde H}
\newcommand{\BZ}{\Omega}
\newcommand{\kk}{\mbf{k}}


\begin{document}
\title{Summary of analytically known tight-binding density of states test cases}
\author{Alex Barnett}
\date{\today}
\maketitle
\begin{abstract}
  We collect known density of states formulae for use as validation
  of numerical Brillouin zone integration methods, and in understanding
  the types of complex-plane singularities that occur.
\end{abstract}

Consider a crystalline quantum system with band Hamiltonian $H(k)$,
with wavevector $\kk\in \BZ := [0,2\pi)^d$, where the dimension $d=1$, $2$, or $3$.
We are handed a band Hamiltonian function in the form of a finite Fourier
series labelled by an integer $m := (m_1,\dots,m_d)$,
\be
H(k) = \sum_{m\in{\cal M}} H_m e^{im\cdot \kk}
\label{Hk}
\ee
where $\cal M := \{(m_1,\dots,m_d): |m_i|\le M\}$ is an array of 
$(2M+1)^d$ frequency indices.
$H$ and each $H_m$ are matrices of size $n$.
$H(\kk)$ is self-adjoint for all $\kk\in\R^d$,
equivalent to the symmetry $H_m^* = H_{-m}$ for all $m$, where
$^*$ indicates the complex transpose (Hermitian conjugate).

We stick to the paradigm self-energy term $\Sigma = iI\eta$
with broadening parameter $\eta\in\R$.
In this case, $\eta$ can be viewed as the imaginary part of
a complex energy $\om = \om_0 + i\eta$ in the closed upper half plane.
Given $M$, $\{H_m\}$ as above, energy $\om_0$, and $\eta\ge0$,
the quantity to compute is
\be
G(\om) = G(\om_0,\eta) \;:=\;
\int_\BZ \tr\, [\om I - H(\kk)]^{-1} d\kk,
\label{G}
\ee
which is sometimes called the Green's function.
We are interested in analytic formulae for $G(\om)$ to
test numerical methods, and to understand
its singularities for $\om$ in the closed upper half plane.

The density of states (DOS) at energy $\om_0$ and broadening $\eta$ is then
\be
A(\om_0) := -\frac{1}{\pi}\im G(\om_0,\eta),
\label{DOS}
\ee
which is thus no longer analytic with respect to $\om=\om_0+i\eta$ in the upper half plane.

\section{Scalar linear chain, square, and simple cubic lattices}

Here $n=1$ (a scalar case). We take fixed
nearest-neighbor tight binding interactions (of $1/2$) which become
the coefficients at frequencies $\pm 1$ in each dimension.
Some have a factor of 2 or $2t$ in the definition (for interaction strength $1$ or $t$) relative to the below.
These results are adapted from \cite[Ch.~5]{economou}
and various papers of the 1970s.


\bfi % ffffffffffffffffffff
\centering
\ig{width=3.5in}{zpaths}    % or use zpaths_jl.png
\ca{Paths of the two roots $z_\pm$ in Section~\ref{s:1d} as
  $\om$ passes along a constant-imaginary-part
  path from very roughly $-2+0.1i$ to $2+0.1i$.}{f:zpaths}
\efi

\subsection{$d=1$ linear chain}
\label{s:1d}

We use $x$ for $\kk$ since it is scalar. Thus $H(x) = \cos x$ and
\bea
G(\om) &=& \int_0^{2\pi} \frac{dx}{\om - \cos x}
= 2i \int_{|z|=1} \frac{dz}{z^2-2\om z + 1}
= 2i \int_{|z|=1} \frac{dz}{(z-z_+)(z-z_-)},
\nonumber
\eea
where we changed variable $z = e^{ix}$ so $dz/z = i\,dx$.
The denominator roots are $z_\pm = \om\pm i\sqrt{1-\om^2}$, where
the sign of the discriminant has been flipped in order to allow
the usual branch cut and square-root definition to apply.
Thus for $\im \om > 0$ the sign of $\im z_\pm$
matches the subscript of $z_\pm$.
Figure~\ref{f:zpaths} shows the resulting paths of $z_\pm$ as $\re \om$
varies at constant $\im \om$.
For $\im \om \ge 0$ then only $z_-$ is inside the unit circle, so
the residue is $2i/(z_--z_+) = -1/\sqrt{1-\om^2}$, so, by
the residue theorem,
\[
G(\om) \;=\; \frac{2\pi}{i \sqrt{1-\om^2}}.
\]
Considering $\om$ real ($\eta=0^+$),
$G$ is pure negative real for $\om<-1$ (below the band),
pure negative imaginary for $\om \in (-1,1)$ (in the band),
then pure positive real for $\om>1$ (above the band).
See \cite[Fig.~5.6]{economou}.
From \eqref{DOS} the DOS is $A(\om) = 2/\sqrt{1-\om^2}$ in the band and zero
outside the band.
For $\im \om$ small and positive (broadening),
the phase of $G(\om)$ rotates rapidly from real to imaginary as
$\re\om$ increases through $-1$, and vice versa at $1$.
The magnitude peak is roughly $\sqrt{2/\eta}$.
For $|\om|\gg 1$, $G \sim 2\pi/\om$ decays as if from
a single pole (which the two inverse-sqrt singularities combine to
behave like at large distances).


\subsection{$d=2$ square lattice}
\label{s:2d}





\bibliographystyle{abbrv}
\bibliography{refs}

\end{document}
