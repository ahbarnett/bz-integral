\documentclass[11pt]{article}

\textwidth 6.5in
\oddsidemargin=0in
\evensidemargin=0in
\textheight 9in
\topmargin -0.5in

\usepackage{graphicx,bm,amssymb,amsmath,amsthm}

\usepackage{showlabels}

\newcommand{\bi}{\begin{itemize}}
\newcommand{\ei}{\end{itemize}}
\newcommand{\ben}{\begin{enumerate}}
\newcommand{\een}{\end{enumerate}}
\newcommand{\be}{\begin{equation}}
\newcommand{\ee}{\end{equation}}
\newcommand{\bea}{\begin{eqnarray}} 
\newcommand{\eea}{\end{eqnarray}}
\newcommand{\ba}{\begin{align}} 
\newcommand{\ea}{\end{align}}
\newcommand{\bse}{\begin{subequations}} 
\newcommand{\ese}{\end{subequations}}
\newcommand{\bc}{\begin{center}}
\newcommand{\ec}{\end{center}}
\newcommand{\bfi}{\begin{figure}}
\newcommand{\efi}{\end{figure}}
\newcommand{\ca}[2]{\caption{#1 \label{#2}}}
\newcommand{\ig}[2]{\includegraphics[#1]{#2}}
\newcommand{\tbox}[1]{{\mbox{\tiny #1}}}
\newcommand{\mbf}[1]{{\mathbf #1}}
\newcommand{\half}{\mbox{\small $\frac{1}{2}$}}
\newcommand{\vt}[2]{\left[\begin{array}{r}#1\\#2\end{array}\right]} % 2-col-vec
\newcommand{\mt}[4]{\left[\begin{array}{rr}#1&#2\\#3&#4\end{array}\right]} % 2x2
\newcommand{\eps}{\varepsilon}
\newcommand{\bigO}{{\mathcal O}}
\newcommand{\sfrac}[2]{\mbox{\small $\frac{#1}{#2}$}}
\newcommand{\R}{\mathbb{R}}
\newcommand{\C}{\mathbb{C}}
\newcommand{\N}{\mathbb{N}}
\newcommand{\Z}{\mathbb{Z}}
\DeclareMathOperator{\re}{Re}
\DeclareMathOperator{\im}{Im}
\DeclareMathOperator{\tr}{Tr}
\DeclareMathOperator{\res}{res}
\newtheorem{thm}{Theorem}
\newtheorem{lem}[thm]{Lemma}
\newtheorem{alg}[thm]{Algorithm}
\newtheorem{pro}[thm]{Proposition}
\newtheorem{cor}[thm]{Corollary}
\newtheorem{rmk}[thm]{Remark}
\newtheorem{conj}[thm]{Conjecture}
% this work...
\newcommand{\om}{\omega}
\newcommand{\tH}{\tilde H}


\begin{document}
\title{Faster numerical computation of 1D Brillouin zone integrals in quantum physics}
\author{Alex Barnett, Jason Kaye, Lorenzo van Mu\~noz}
\date{\today}
\maketitle
\begin{abstract}
  Notes on methods for numerical approximation of density of states
  (DOS) and related integrals, in the 1D case, where the Hamiltonian
  is given by a Fourier series. They could be used
  to perform the innermost integral in IAI (iterated adaptive integration)
  from our autobz paper.
  Both scalar and small matrix cases are considered.
  This accompanies Julia benchmarking code in \texttt{contdef1d}.
\end{abstract}

Consider a crystalline quantum system with band Hamiltonian $H(k)$.
Let $k$ denote the $2\pi$-periodic wavevector (a scalar since we are in 1D), and
the Brillouin zone we may take as any periodic interval, eg, $[0,2\pi)$.
We are given the band Hamiltonian function in the form of a finite Fourier
series
\be
H(k) = \sum_{m=-M}^M H_m e^{imk}
\label{Hk}
\ee
which has $2M+1$ terms. $H$ and each $H_m$ are matrices of size $n$
(given in some Wannier basis that we need not be concerned about here).
We refer to the case $n=1$ as scalar.
For CCQ applications, $n$ is not large.
In the application, $H(k)$ is self-adjoint for all $k\in\R$,
equivalent to the symmetry $H_m^* = H_{-m}$ for all $m$, where
$^*$ indicates the complex transpose (Hermitian conjugate).

Given $M$, $\{H_m\}$ as above, $\om\in \R$, and $\eta\ge0$,
the paradigm integration task is to evaluate
\be
A = A(\om,\eta) := \int_{0}^{2\pi} \tr\, [(\om +i\eta)I - H(k)]^{-1} dk
\label{A}
\ee
The inverse may be interpreted as a Green's function, and $\eta$
a possible broadening due to temperature, disorder, interactions, etc.
The density of states at energy $\om$ is then
\be
\rho(\om) = -\frac{1}{\pi}\im A(\om,\eta)~.
\label{DOS}
\ee
% *** Tr and Im commute, but conceptuially shoudl the Tr be outermost?
The above is the simple case of constant matrix $(\om+i\eta)I$; this
can be a more general non-Hermitian matrix often denoted by
$\Sigma_\om$, still independent of $k$.
% although if not, that would just modify the F series coeffs for H.

The ``conventional'' numerical method for \eqref{A} is
to apply adaptive Gauss--Kronrod integration on the real axis.
This was proposed for the innermost integral in \cite{autobz}.
When $\eta\ll1$ and $\om$ is in a band (meaning there is as least one
$k\in\R$ where $H(k)=\om$), this adaptive scheme is expected
to refine to a $k$-scale of about $\eta$, giving a $\log(1/\eta)$ cost.

We also care about the limit $\eta\to0^+$, which cannot be directly
computed by adaptive real-axis quadrature. It could be accessed by
Richardson extrapolation from larger $\eta$ values, but we have not
studied how automated that could be.


% sssssssssssssssssssssssssssssssssssssssssssssssssssssssssssssssssssssssssss
\section{Residue method for the scalar case ($n=1)$}
\label{scalar}

Changing variable to $z=e^{ik}$,
so that $dz = i z dk$,
in the scalar case \eqref{A} becomes an integral
\be
A = \int_{0}^{2\pi} \frac{dk}{\om +i\eta - H(k)}
 = \int_{|z|=1} \frac{1}{\om +i\eta - \tH(z)} \, \frac{dz}{iz}.
\label{As}
\ee
Here $\tH:\C\to\C$ is the analytic continuation of the
Hamiltonian function in terms of $z$,
defined by
\be
\tH(e^{ik}) = H(k) \qquad \mbox{ for all } k\in\C,
\label{tH}
\ee
where we have used the analytic continuation of $H(k)$
as an entire function throughout $\C$ defined by its series \eqref{Hk}.
Note that $\tH$ is the Laurent series
\be
\tH(z) = \sum_{m=-M}^M H_m z^m,
\label{Laur}
\ee
is real-valued on the unit circle, and, when $\om$ is in a band,
some poles in the above integrand will approach the unit circle
as $\eta\to0^+$.

We apply Cauchy's residue theorem to the final integral \eqref{As}.
There are two cases for
the integrand
\be
f(z) := [iz(\om+ i\eta - \tH(z))]^{-1}.
\label{f}
\ee
Either $H\equiv c$ is constant, in which case the simple pole at the
origin gives
$A = 2\pi / (\om+i\eta-c)$, or $H$ is non-constant, so that
$H_m\neq 0$ for some $m<0$ (keeping in mind Hermitian symmetry).
The first case is trivial, so from now we deal with the second case.
Although $f$ is not defined at $z=0$ by the above, the following shows that
it may be extended to this point to give a locally analytic function.
\begin{pro}  % ppppppppppppppppppppppppppppppppppppppppppppppppppppppppppppp
  \label{p:originok}
  Let $H$ be a non-constant Fourier series \eqref{Hk}, then
  the integrand $f$ defined by \eqref{f}, where $\tH$ is defined by
  \eqref{tH}, has a removable singularity at $z=0$.
\end{pro}   % ppppppppppppppppppppppppppppppppppppppppppppppppppppppppppppppp
\begin{proof}
  Let $p \in \{1,2, \dots, M\}$
  be the largest index $m$ for which $H_{-m} \neq 0$.
  Inserting \eqref{Laur} gives
  $$
  f(z) = \frac{1}{iz}\left( H_{-p}z^{-p} + H_{-p+1}z^{-p+1} + \dots \right)^{-1}
  =
  \frac{z^{p-1}}{iH_{-p}}\left( 1 + \frac{H_{-p+1}}{H_{-p}}z + \dots \right)^{-1}
  $$
  where the dots indicate a finite sum of higher powers of $z$.
  This defines an extension of $f$ analytic at $z=0$.
\end{proof}
In particular this means that the residue theorem may be applied to
\eqref{As}, with no further concern given to the origin.
Thus, when $\eta>0$, so that no zero of $\om + i\eta - \tH(z)$ can lie
on the unit circle, poles lie either inside or outside of this circle.
Assuming all poles are simple,
\be
A = \int_{|z|=1} f(z) dz = 2\pi i \sum_{|z_j|<1} \res_{z_j} f
\ee
where each residue can be evaluated by
\be
\res_{z_j} f = \lim_{z\to z_j} (z-z_j)f(z) = \frac{i}{z_j \tH'(z_j)}
\label{res}
\ee
where $\tH'$ is the complex derivative of the Laurent series \eqref{tH}.
Since
\be
iz\tH'(z) = H'(i^{-1}\log z),
\label{deriv}
\ee
the residue can
also be written
\be
\res_{z_j} f  = \frac{-1}{H'(i^{-1}\log z_j)}
\ee
in terms of the complex wavevector derivative $H'$ of the original band
Hamiltonian \eqref{Hk}.

Numerically roots of the denominator $iz(\om + i\eta - \tH(z))$
are found as follows.
The set of such nonzero roots is the same as the nonzero roots of
$$
z^M(\om + i\eta - \tH(z)),
$$
a polynomial of degree $2M$ with coefficients
$a_j = (\om+i\eta)\delta_{jM} - H_{j-M}$ for $j=0,1,\dots,2M$.
This ``Laurent to Taylor'' conversion
trick is due to J.\ Boyd in the context of root-finding and used in, eg,
Chebfun.
One finds its roots by either the companion matrix
eigenvalues or iterative deflation with Laguerre's method.
Currently we use an $\bigO(M^2)$ scaling fast implementation of the former
(\cite{aurentz1,aurentz2} as coded in \texttt{AMRVW.jl})
for degrees greater than 200, else the
latter (\cite{skowron} as coded in \texttt{PolynomialRoots.jl}).
The latter is around 10 times faster for the same degree, but
appears unstable for degrees above about 200,
or (update) sometimes even for degrees above 60.
Zero roots (which may arise for vanishing extremal coefficients
$H_{\pm M}$, $H_{\pm (M-1)}$, etc) are discarded since it has
already been shown that $f$ is analytic at the origin.

Finally we consider
the case $\eta\to0^+$: the denominator then often has roots lying on $|z|=1$,
corresponding to real band intersections (Fermi ``points'')
$k$ where $H(k)=\om$.
Only some of the corresponding poles in the integrand $f(z)$ of \eqref{f}
should be included in the residue theorem.
Denominator roots whose location approaches the unit circle from the inside
are included, otherwise they are excluded.
Writing $z_0(\eta)$ for the $\eta$-dependence of such a root,
taking the total $\eta$-derivative of $(\om+i\eta)-\tH(z_0(\eta)) = 0$
gives a formula $z_0' = dz_0/d\eta = i/\tH'(z_0)$.
Approaching from the inside means geometrically
$\re z_0' \overline{z_0}<0$, recalling
that $\eta$ is \textit{decreasing} to $0$.
Substituting $z_0'$ gives the criterion for inclusion
$\re i \overline{z_0}/\tH'(z_0) <0$, or, using \eqref{deriv}
and $z_0\overline{z_0} = 1$,
$$
\re H'(k_0) > 0,
$$
where $k_0 = i^{-1} \log z_0$ is the real-valued wavevector corresponding to
the unit circle root $z_0$.
This has a simple interpretation that only up-going band intersections
should be included (where pushing $\om$ into a small imaginary
part causes the intersection to move up from the real axis, and its
image inside the unit circle).

A good numerical check for the sign is the simple test case band structure
$H(k) = \cos k$, whose only nonzero coefficients are $H_{\pm1} = 1/2$,
for which the exact (positive!) DOS is easily found by change of variable,
$$
\rho(\om) = \int_0^{2\pi} \delta(H(k)-\om) dk
= \sum_{k\in[0,2\pi]: H(k)=\om} \frac{1}{|H'(k)|}
= \frac{2}{\sin(\cos^{-1} \om)} = \frac{2}{\sqrt{1-\om^2}}
$$
in the band $|\om|<1$, and zero for $|\om|>1$.
The 1D Van Hove singularity is undefined at the band edge $\om=\pm1$.

We implement the scalar case for $\eta>0$ and $\eta=0^+$
by the method \texttt{discresi} in \texttt{Cont1DBZ}.



*** Need to discuss/research/test: nonsimple poles inside, and on unit circle
for $\eta=0^+$.




% mmmmmmmmmmmmmmmmmmmmmmmmmmmmmmmmmmmmmmmmmmmmmmmmmmmmmmmmmmmmmmmmmmmmmmmmmm
\section{Residue method for the matrix case $n>1$}

% Beyn 2012 LAA, Sec 2 reviews Keldysh. *** need better math cites

The main tool is the matrix version of the residue theorem,
summarized in \cite[Thm.~2.9]{beyn12}.
Let $\Omega \subset \C$ be open, and let
$B: \Omega \to \C^{n\times n}$ be an analytic matrix-valued function.
The nonlinear eigenvalues of $B$ in $\Omega$ are the distinct
countable $z_j\in\Omega$, $j=1,\dots,J$, such that
$B(z_j)v_j = 0$ for each nontrivial (right) eigenvector $v_j\in\C^n$.
Each also has a left eigenvector $u_j$ such that $u_j^*B(z_j) = 0$.
We leave their normalization free.
Then, in the case of all eigenvalues simple and none lying on $\partial\Omega$,
\be
\int_{\partial\Omega} B(z)^{-1} dz =
2\pi i \sum_{j=1}^J \res_{z_j} B^{-1},
\quad \mbox{ where }
\quad
\res_{z_j} B^{-1} := \lim_{z\to z_j} (z-z_j)B(z)^{-1}
= \frac{v_j u_j^*}{u_j^* B'(z_j) v_j},
\label{resmat}
\ee
where as usual $B'=dB/dz$.
This is an analytic generalization of the spectral projector for the linear
case $B(z) = B_0 + B_1z$.
Note that the definition of the residue is identical
to \eqref{res} replacing $f$ by the matrix-valued meromorphic
function $B^{-1}$; each residue is a rank-1 matrix,
with a derivative in the denominator as in \eqref{res}.
The nonsimple case relating the higher order pole
coefficients to the generalized eigenvectors of $F(z)$
is more complicated and due to Keldysh, Gohberg \cite{gohberg}, etc.
For now we assume all eigenvalues are simple.

We apply this Keldysh residue theorem to
$$
B(z) = iz[(\om+i\eta)I - \tH(z)],
$$
since \eqref{A}, after changing to $z=e^{ik}$, is then
$$
A = \int_{|z|=1} B(z)^{-1} dz.
$$
By analogy with Prop.~\ref{p:originok}, for a non-constant $\tH(z)$, $B^{-1}$
is holomorphic at the origin (*** this should be fleshed out).

The steps are:
i) find all nonlinear eigenvalues $z_j$ lying in the unit disc
for the polynomial $z^MB(z)$,
which is a polynomial eigenvalue problem (PEP);
ii) extract all right and left eigenvectors and $B'$ at each eigenvalue;
iii) compute all residues (at this point $\tr$ may be taken);
iv) sum as in \eqref{resmat}.
For the case $\eta=0^+$, any eigenvalues on the unit circle
but which lay in the unit disc for $\eta>0$ must be appended to the list.



\section{Timing results}

We want to know the speed-up factor of above residue (R) method vs
conventional adaptive (A) quadrature on the real axis.
Call this
factor ${\cal F} := t_\tbox{A}/t_\tbox{R}$.
Stick to single-thread Julia and same for LAPACK libraries called.
(A) is pretty optimal due to LXVM optim.

Fix default parameters:
\ben
\item
  $\eta=10^{-5}$ (the smallest relevant in CCQ physics).
  Since $\log_2(\eta) \approx 17$, this demands about 35 extra panels per band
  intersection.

  There is also some intellectual interest in being able to take
  $\eta=0^+$, although this could be done via Richardson from selected
  values eg $\eta = [1,2,4]\times 10^{-3}$, if such values are smaller than
  other length-scales in the problem (this may be impossible near band edges).
\item
  $\eps = 10^{-5}$ tolerance for (A). However, (R) usually gets accuracy
  close to machine precision,
  so, hard to know what to fairly compare here.
\item Ryzen 2nd-gen 5700U laptop, 8-core, CPUMark score 2640 (1 thread),
  using AC power.
\een
  
The main uncertainty in the band structure setting is the
``number of band intersections ($N_\tbox{BI}$, each giving a pole near real axis) per Fourier mode'',
which we call ${\cal Z}$.
Eg if $M=10$ and randn coeffs are used, there's around 14 zeros,
so ${\cal Z} = 0.7$.
But if exp decaying
coeffs used (we pick a decay of $e^{-0.5|m|}$ which goes down to
0.006 at $m=M=10$),
${\cal Z} = 0.2$ (typ 2--4 zeros; it seems silly to push down
to a median of 0 zeros, where (A) obviously wins).


\subsection{Scalar case}

The cost of (A) is $\bigO(M N_\tbox{BI} \log \eta^{-1})$,
since there is adaptive quad down to $\eta$ scale around each band intersection
(BI), and the $M$ factor is the Fourier series evaluation.

The cost of (R) is $\bigO(M^2)$, due to using the fast companion EVP
method of \cite{aurentz1,aurentz2} or the deflation rootfinder
of \cite{skowron}.
It is indep of $\eta$ or number of BI.

(R) takes 24 $\mu$s (exploiting the deflation rootfinder)
or 140 $\mu$s (for fast companion).

(A) takes about 35 $\mu$s per BI. This gives
450 $\mu$s for randn coeffs, but only 70 $\mu$s for $N_\tbox{BI}=2$.

So ${\cal F} \approx 20$ for randn
but only ${\cal F} \approx 3.5$ for decaying coeffs.

\subsection{Matrix case}

For general $n$,
the cost of (A) is $\bigO((n+M)n^2 N_\tbox{BI} \log \eta^{-1})$,
where the first term is from the $\bigO(n^3)$ matrix inverses,
and the 2nd from the Fourier series evaluation of matrix entries.

The cost of (R) is $\bigO(n^2M^2)$ in theory, to find
all PEP eigenpairs, using deflation, some as-yet-unknown
block fast companion scheme, or the fast scalar companion on
$\det B(z)$ which is a degree $nM$ polynomial.

So for $N_\tbox{BI}\approx 1$, and $n\approx M$,
(R) is quartic but (A) only cubic in the size.
But (A) has the $\log 1/\eta$ extra factor.

Here we chose a sensible small $n=5$ typical of some CCQ apps.

(R) costs 23 ms using the library PEP solver via naive block companion,
at $\bigO((nM)^3)$.
This is a size $2nM=100$ generalized EVP since the companion matrix is
a pencil in {\tt NonlinearEigenproblems.jl}, as in Merhmann--Voss.
Even simply using plain EVP would be twice as fast.

(A) costs 1 ms per BI, as predicted by $n^2=25$ times the scalar cost for (A).
For randn matrix coeffs, this gives 17 ms.
For decaying coeffs (2 BIs), only 2 ms.

So, if $N_\tbox{BI}=2$ only, (R) is 10x slower than (A),
${\cal F} = 0.1$ and we do not win.

My memory of bands is that there are often more than 2 BIs in each
1D integral, maybe 4 or 6. But we still do not win, with this current
PEP solver.

We have not tried a $\bigO((nM)^2)$ PEP solver yet,
however the fast scalar companion takes 10 ms for degree $2nM=200$;
this is not much different from the existing PEP solver at 20 ms.

Upshot is we're wasting time finding 100 roots inside the circle,
so that only a few problematic zeros near unit circle can be removed.
It is not worth it. This suggests returning to contour deformation/shift.

*** To do: more serious benchmarking of all types of PEP solvers.

\subsection{Issues / Discussion}

\ben
\item Failure for exp decaying coeffs at large $M$ (eg 100).
  There are unstable roots far from the unit circle (inside and outside),
  forming annuli.
  This problem doesn't happen with randn coeffs since there almost all zeros
  are on the unit circle, and stably computed.
  Jason anticipated something like this.
  It suggests onyl using contour deformations or shifts, plus finding
  PEP eigenvalues only in the shifted-over region via deflation.
  
\item Need {\tt quadgk} to output what nodes or panel breakpoints it used,
  number of func evals, etc.
  Without that it is hard to fight it using timings alone.

\item Find some zeros (or PEP eigenpairs) vs all of them?
  
  Residue method needs all the polynomial zeros inside the unit circle,
  which is about half the total, ie, is about $M$.
  Thus an all-zeros method (such as companion) seems best, even for
  the matrix case; it only wastes a factor of 2.
  But it's a big waste to need all zeros in the circle this way.
  
  A contour-deformation method only needs nearby zeros, which scales
  like $N_\tbox{BI}$, so may be 10x less than $M$.

\item
  Band-edges: near band edges (A) gets easier since the poles are only
  $\bigO(\sqrt{\eta})$ from the real axis, no longer $\bigO(\eta)$.
  It may be Richardson breaks here.

  The graphene $n=2$ exact DOS is a good test for band edges and Van Hove.

\item Robustness studies; this has to not break over billions of calls.
\een


% To do

% * apply exp decay to randn F coeffs to get far away poles, chk no issues

% * use Bindel-Hood winding Rouche to count PEP eigenvalues in disc?

% * local quadgk panel perturbation into C plane iff
% a local lin model fits suff well, for a panel already small enough.
% The local lin model has to say which side the root is to deform either
% up or down.
  


\bibliographystyle{abbrv}
\bibliography{refs}

\end{document}
